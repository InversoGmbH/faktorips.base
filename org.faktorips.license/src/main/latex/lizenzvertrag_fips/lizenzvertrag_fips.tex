\documentclass[10pt]{article}
\usepackage{geometry}
\usepackage{cmap}
\usepackage[utf8]{inputenc}
\usepackage{ngerman}
\usepackage[T1]{fontenc}
\usepackage{lmodern}

\geometry{a4paper,left=30mm,right=30mm, top=30mm, bottom=30mm} 

\setlength{\parskip}{1.2ex}
\setlength{\parindent}{0em}

\renewcommand{\familydefault}{cmss}

\makeatletter
\renewcommand{\section}{\@startsection{section}{1}{\z@}{3ex}{1ex}{\normalfont\large\bfseries}}
\renewcommand{\subsection}{\@startsection{subsection}{2}{\z@}{1ex}{1ex}{\normalfont\normalsize\bfseries}}
\makeatother

\begin{document}

\title{Lizenzvereinbarung Faktor IPS\\
für die Version \input{version.txt}}
\author{zwischen dem Lizenznehmer und der \\
Faktor Zehn AG, Neumarkter Straße 71, 81673 München, als Lizenzgeber}
\date{\today}
\maketitle

\section*{Präambel}

Faktor Zehn ist Anbieter des Softwareprodukts ``Faktor-IPS'' (im Folgenden ``Vertragssoftware'' genannt). Faktor Zehn
ist Inhaber der Rechte an den proprietären Komponenten von Faktor IPS. Zusammen mit diesen proprietären Komponenten wird
auch die in der \textbf{Anlage \ref{sec:drittsoftware}} bezeichnete Drittsoftware überlassen (im Folgenden
„Drittkomponenten“); die Rechtsinhaberschaft und Rechtseinräumung an diesen Drittkomponenten richtet sich ausschließlich
nach den in der \textbf{Anlage \ref{sec:drittsoftware}} bezeichneten Lizenzbedingungen.

Der Lizenznehmer beabsichtigt, die Vertragssoftware selbst zu nutzen bzw. zusammen mit eigener oder Drittsoftware seinen
Kunden zur nicht ausschließlichen Nutzung zu überlassen.

Faktor Zehn überlässt die Vertragssoftware an Dritte auf Basis eines spezifischen ``Open-Source''-Modells, welches in
dieser Lizenzvereinbarung definiert ist. Die Vertragspartner sind sich darüber einig, dass die hier vereinbarte
``Open-Source''-Lizenz nicht der Definition der Open Source Initiative (OSI, http://www.opensource.org) entspricht und
keine Rechte und Pflichten aus dieser Definition oder aus anderen Opern-Source Modellen im Hinblick auf die vorliegende
Lizenzvereinbarung ableitbar sind. 

Vor diesem Hintergrund vereinbaren die Parteien folgende Bedingungen zur Überlassung und Nutzung der Vertragssoftware.

\section{Anwendungsbereich dieser Lizenzvereinbarung}

\subsection{Geltung für die Aktuelle Version der Vertragsoftware}
Diese Vereinbarung gilt für die Überlassung und Nutzung der vorstehend in der Überschrift bezeichneten Version des
Softwareprodukts Faktor IPS. Der Lizenznehmer kann diese Version von der Website des Lizenzgebers (URL
http://www.faktorzehn.org) per Online-Download herunterladen.

\subsection{Künftige Versionen der Vertragssoftware}
Der Lizenzgeber behält sich vor, die auf der Website zum Download bereitgestellten Versionen des Softwareprodukts
Faktor-IPS jederzeit gegen aktualisierte Versionen auszutauschen. Die Überlassung und Nutzung von aktualisierten
Versionen kann jeweils gesonderten Lizenzbedingungen unterstellt sein, die von den vorliegenden Lizenzbedingungen
abweichen.

Der Lizenzgeber kündigt die bevorstehende Ablösung einer Version der Vertragssoftware durch eine aktualisierte Version
im Regelfall durch einen Hinweis auf seiner Website an.

\subsection{Abrufbarkeit der Lizenzvereinbarung}
Diese Lizenzvereinbarung ist vor dem Download der vorstehend genannten Version der Vertragssoftware auf der Website des
Lizenzgebers aufrufbar und kann vom Lizenznehmer gespeichert und ausgedruckt werden. Des Weiteren kann der Text dieser
Lizenzvereinbarung vom Lizenznehmer zusammen mit der Vertragssoftware als PDF-Dokument heruntergeladen werden. 

\subsection{Einverständnis des Lizenznehmers mit der Geltung dieser Lizenzvereinbarung}
Der Lizenznehmer erklärt sein Einverständnis mit der Geltung dieser Lizenzvereinbarung durch Betätigung der 
Zustimmungs-Schaltfläche auf der Website des Lizenzgebers.

\subsection{Keine Geltung von Allgemeinen Geschäftsbedingungen des Lizenznehmers}
Allgemeine Geschäftsbedingungen des Lizenznehmers finden keine Anwendung, soweit sie von den Bestimmungen dieser
Nutzungsvereinbarung abweichen oder diesen entgegenstehen.

\section{Vertragsgegenstand}

\subsection{Überlassung und Rechtseinräumung}
Faktor Zehn überlässt dem Lizenznehmer die Vertragssoftware in der Oben bezeichneten Version und im gegebenen
Entwicklungsstand im Wege des Download und räumt dem Lizenznehmer hieran Nutzungsrechte gemäß Ziffer 3 ein.

\subsection{Unentgeltlichkeit}
Die Überlassung der Vertragssoftware sowie die Rechtseinräumung erfolgt unentgeltlich.

\subsection{Umfang der Vertragssoftware}
Gegenstand der Überlassung ist der Quellcode („Open-Source“ im Sinne dieser Lizenzvereinbarung) der Vertragssoftware
inklusive, soweit vorhanden, zugehöriger Kompilate.

\subsection{Funktionsumfang}
Der Funktionsumfang der Vertragssoftware ist durch den Quellcode im überlassenen Stand definiert. Eine gesonderte
Dokumentation wird nicht zur Verfügung gestellt.

\subsection{Funktionalitäten, Qualitätsmerkmale}
Die Vertragssoftware wird im vorliegenden Entwicklungsstand überlassen; eine bestimmte Qualität oder die
vollständige Mängelfreiheit werden nicht zugesagt. Besondere Funktionalitäten, eine besondere Funktionsfähigkeit, die
Kompatibilität mit Drittprodukten oder sonstige Merkmale der Vertragssoftware sind nicht geschuldet.

\subsection{Weiterentwicklung, Pflege}
Die Weiterentwicklung und Verbesserung der Vertragssoftware liegt im freien Ermessen des Lizenzgebers. 

\section{Rechtseinräumung}

\subsection{Proprietäre Elemente der Vertragssoftware}
\label{sec:nutzungsrecht}
Faktor Zehn räumt dem Lizenznehmer an den proprietären Komponenten der Vertragssoftware das einfache, nicht
ausschließliche und zeitlich unbeschränkte Recht ein, diese sowohl im Quellcode als auch in kompilierter Form - sofern
diese gleichzeitig zur Verfügung gestellt wurde - für dessen eigene Zwecke zu nutzen.

Das Nutzungsrecht umfasst das Recht, die proprietären Komponenten der Vertragssoftware in dem zur vertragsgemäßen
Nutzung erforderlichen Umfang zu vervielfältigen, sowie das Recht, diese zu bearbeiten und weiterzuentwickeln, sofern
hierbei jeweils die in der Vertragssoftware vorhandenen Autorenbezeichnungen, Schutzrechtsvermerke und sonstigen
Hinweise, die auf die Rechtsinhaberschaft der Faktor Zehn oder Dritter verweisen, nicht verändert oder entfernt werden.

\subsection{Drittkomponenten}
Die Einräumung von Nutzungsrechten an Drittkomponenten erfolgt jeweils ausschließlich nach Maßgabe der in der
\textbf{Anlage \ref{sec:drittsoftware}} für die entsprechenden Drittkomponenten bezeichneten Lizenzbestimmungen.

\subsection{Rechtseinräumung an Dritte}
Der Lizenznehmer ist berechtigt, die Vertragssoftware an einen Dritten zu überlassen und diesem hieran Nutzungsrechte
hinsichtlich proprietärer Komponenten entsprechend der vorstehenden Ziffer \ref{sec:nutzungsrecht} einzuräumen. Der
Lizenznehmer verpflichtet sich, den Lizenzgeber von dieser Überlassung an den Dritten zu informieren. 

Die Überlassung von Drittkomponenten und die Rechtseinräumung hieran bestimmt sich ausschließlich nach den in der
\textbf{Anlage \ref{sec:drittsoftware}} aufgeführten Lizenzbestimmungen.

\section{Empfehlungen des Lizenzgebers}

Faktor Zehn empfiehlt dem Lizenznehmer, die Tauglichkeit der Vertragssoftware für die von ihm verfolgten Zwecke
sorgfältig zu prüfen. Des Weiteren wird empfohlen, die Vertragssoftware daraufhin zu überprüfen, ob sie für die vom
Lizenznehmer verwendete IT-Infrastruktur tauglich ist, insbesondere im Hinblick auf die Kompatibilität mit anderer vom
Lizenznehmer verwendeter Drittsoftware.

Faktor Zehn empfiehlt des Weiteren, vor erstmaligem Einsatz der Vertragssoftware sowie während der Nutzung der
Vertragssoftware in angemessenen Zeiträumen Sicherungen von Daten vorzunehmen, um einem möglichen Datenverlust
vorzubeugen.

\section{Haftung für Mängel und Sonstige Haftung}

\subsection{Haftung für Rechtsmängel}
\label{sec:haftung_rechtsmangel}
Verschweigt der Lizenzgeber arglistig einen Mangel im Recht der Vertragssoftware, so ist er verpflichtet, dem
Lizenznehmer im Rahmen der Gesetze den daraus entstehenden Schaden zu ersetzen. Der Lizenznehmer hat den Lizenzgeber
unverzüglich von geltend gemachten Rechten Dritter zu benachrichtigen und den Lizenzgeber bei der Rechtsverteidigung im
Rahmen des Zumutbaren zu unterstützen.

\subsection{Haftung für Sachmängel}
\label{sec:haftung_sachmangel}
Verschweigt der Lizenzgeber arglistig einen Fehler der Vertragssoftware, so ist er verpflichtet, dem Lizenznehmer im
Rahmen der Gesetze den daraus entstehenden Schaden zu ersetzen.

\subsection{Allgemeine Haftung; Produkthaftung}
Der Lizenzgeber haftet unbeschadet der vorstehenden Ziffern \ref{sec:haftung_rechtsmangel} und
\ref{sec:haftung_sachmangel} im Rahmen der Gesetze nur für Vorsatz und grobe Fahrlässigkeit. Eine Haftung nach Maßgabe
des Produkthaftungsgesetzes bleibt unberührt.

\subsection{Haftungsausschluss}
Im Übrigen ist eine Haftung des Lizenzgebers auf Schadensersatz oder Ersatz vergeblicher Aufwendungen ohne Rücksicht auf
die Rechtsnatur des Anspruchs ausgeschlossen. Dies gilt auch zugunsten von Mitarbeitern und sonstigen Erfüllungsgehilfen
des Lizenzgebers.

\section{Schlussbestimmungen}

\subsection{Anwendbares Recht}
Für alle Streitigkeiten aus oder im Zusammenhang mit dieser Lizenzvereinbarung sowie für deren Auslegung gilt
ausschließlich deutsches Recht unter Ausschluss der Rechtsnormen, die in eine andere Rechtsordnung verweisen; die
Anwendung des UN-Übereinkommens über Verträge über den internationalen Warenkauf ist ausgeschlossen.

\subsection{Gerichtsstand}
Ausschließlicher Gerichtsstand für Streitigkeiten aus oder in Zusammenhang mit diesem Vertrag ist für beide
Vertragsparteien München.

\subsection{Schriftform}
Änderungen und Ergänzungen dieses Vertrages bedürfen der Schriftform. Dies gilt auch für Änderungen dieser Bestimmung.

\subsection{Salvatorische Klausel}
Sollten einzelne Bestimmungen dieses Vertrages ganz oder teilweise nicht rechtswirksam oder nicht durchführbar sein oder
werden oder eine Lücke enthalten, so wird die Gültigkeit der übrigen Bestimmungen dieses Vertrages dadurch nicht
berührt. Anstelle der unwirksamen, undurchführbaren oder fehlenden Bestimmung soll eine angemessene Regelung gelten,
die, soweit rechtlich möglich, dem am nächsten kommt, was die Vertragspartner gewollt hätten, sofern sie bei Abschluss
des Vertrages diesen Punkt bedacht hätten. 


\begin{appendix}
\section{Drittsoftware}
\label{sec:drittsoftware}

\begin{tabular}{|l|l|}
\hline
\bfseries Bezeichnung Drittsoftware & \bfseries Anzuwendende Bedingungen\\\hline

log4j & Apache License, Version 2.0\\
http://www.apache.org & http://www.apache.org/licenses/LICENSE-2.0\\\hline

POI & Apache License, Version 2.0\\
http://www.apache.org & http://www.apache.org/licenses/LICENSE-2.0\\\hline

commons-lang & Apache License, Version 2.0\\
http://www.apache.org & http://www.apache.org/licenses/LICENSE-2.0\\\hline

commons-logging & Apache License, Version 2.0\\
http://www.apache.org & http://www.apache.org/licenses/LICENSE-2.0\\\hline

bsh & Sun Public License (SPL); oder:\\
http.//www.beanshell.org & Gnu Lesser License (LGPL)\\\hline

Groovy & Apache License, Version 2.0\\
http://www.apache.org & http://www.apache.org/licenses/LICENSE-2.0\\\hline

OpenEJB JavaEE-API & Apache License, Version 2.0\\
http://www.apache.org & http://www.apache.org/licenses/LICENSE-2.0\\\hline

OpenCSV & Apache License, Version 2.0\\
http://www.apache.org & http://www.apache.org/licenses/LICENSE-2.0\\\hline

\end{tabular}

\end{appendix}


\end{document}
